%-------------------------
% Resume in Latex
% Original author : Sourabh Bajaj
% Adaptation : Hyunggi Chang (changh95)
% License : MIT
%------------------------

\documentclass[letterpaper,11pt]{article}

\usepackage{latexsym}
\usepackage{kotex} % 한글 사용 가능! 
\usepackage[empty]{fullpage}
\usepackage{titlesec}
\usepackage{marvosym}
\usepackage[usenames,dvipsnames]{color}
\usepackage{verbatim}
\usepackage{enumitem}
\usepackage[hidelinks]{hyperref}
\usepackage{fancyhdr}
\usepackage[english]{babel}
\usepackage{tabularx}
\usepackage{amsmath}

\pagestyle{fancy}
\fancyhf{} % clear all header and footer fields
\fancyfoot{}
\renewcommand{\headrulewidth}{0pt}
\renewcommand{\footrulewidth}{0pt}

% Adjust margins
\addtolength{\oddsidemargin}{-0.5in}
\addtolength{\evensidemargin}{-0.5in}
\addtolength{\textwidth}{1in}
\addtolength{\topmargin}{-0.5in}
\addtolength{\textheight}{1.0in}

\urlstyle{same}

\raggedbottom
\raggedright
\setlength{\tabcolsep}{0in}

% Sections formatting
\titleformat{\section}{
  \vspace{-4pt}\scshape\raggedright\large
}{}{0em}{}[\color{black}\titlerule \vspace{-2pt}]

%-------------------------
% Custom commands - Do Look into this area, if you wish to customise further!
% 포맷이 마음에 들지 않으시다면 이 부분을 수정하세요!

\newcommand{\resumeItem}[1]{
  \item\small{
    {#1 \vspace{-2pt}}
  }
}

\newcommand{\resumeProject}[3]{
  \vspace{0pt}\item
    \begin{tabular*}{0.97\textwidth}[t]{l@{\extracolsep{\fill}}r}
      #1 & \small #2 \\
      {#3}
    \end{tabular*}\vspace{-5pt}
}

\newcommand{\resumeSubProject}[2]{
  \item{
    {#1 \vspace{0pt}}
    {#2}
  }
}

\newcommand{\resumeSubheading}[4]{
  \vspace{-1pt}\item
    \begin{tabular*}{0.97\textwidth}[t]{l@{\extracolsep{\fill}}r}
      \textbf{#1} & #2 \\
      \textit{\small#3} & \textit{\small #4} \\
    \end{tabular*}\vspace{-5pt}
}

\newcommand{\resumeSubItem}[2]{\resumeItem{#1}{#2}\vspace{-4pt}}

\renewcommand{\labelitemii}{$\circ$}

\newcommand{\resumeProjectListStart}{\begin{itemize}[leftmargin=*]}
\newcommand{\resumeProjectListEnd}{\end{itemize}}

\newcommand{\resumeSubProjectListStart}{\begin{itemize}[leftmargin=*]}
\newcommand{\resumeSubProjectListEnd}{\end{itemize}}

\newcommand{\resumeSubHeadingListStart}{\begin{itemize}[leftmargin=*]}
\newcommand{\resumeSubHeadingListEnd}{\end{itemize}}
\newcommand{\resumeItemListStart}{\begin{itemize}}
\newcommand{\resumeItemListEnd}{\end{itemize}\vspace{-5pt}}

%-------------------------------------------
%%%%%%  CV STARTS HERE  %%%%%%%%%%%%%%%%%%%%%%%%%%%%


\begin{document}

%----------HEADING-----------------
\begin{tabular*}{\textwidth}{l@{\extracolsep{\fill}}r}
  \textbf{\href{http://changh95.github.io}{\Large Your Name (경력 기술서)}} & Github: YourGithubID $|$ SlideShare: YourSlideShare \\
  \href{http://changh95.github.io}{http://YourGithubID.github.io} & Email : \href{mailto:YourEmail@gmail.com}{YourEmail@gmail.com} \\
  {} & Mobile : (+82) 010-YOUR-NUMBER
\end{tabular*}

%-----------PROJECTS-----------------
\section{\textbf{1. [초대형 프로젝트 이름]}}
  \resumeProjectListStart
    \resumeProject
    {\textbf{소속}: (주) [회사 이름]} {2020.01 - PRESENT}
    
    {\textbf{주요 내용}:
    
    본 프로젝트는 (주) [회사 이름]의 [제품이름] 제품의 연구 개발을 목적으로 둠. 해당 제품은 다양한 [제품기능] 기능을 제공하며, [기술 스택]를 기반으로 구현되었음. [프로젝트 시기] 멤버로 참여하여 주 업무로 [주 업무 내용]을 맡음.
    
    \textbf{주요 성과}:
    
    본 프로젝트에서 진행한 내용으로는 1. [작업1], 2. [작업2], 3. [작업3]이 있음. 부 업무로는 [부업무1], [부업무2], [부업무3] 등이 있음.}

    \resumeSubProjectListStart
        \resumeSubProject {\textbf{1.1 (예시) 딥러닝 기반 동물 분류 학습}}
        
        {
        50가지가 넘는 동물을 영상을 통해 실시간으로 90\% 이상의 정확도로 분류할 수 있는 시스템을 개발하는 것을 목표로 둠. 구현한 기능들은 아래와 같음. 
        
        \vspace{2mm}
        
        \textbf{1.1.1. 크롤러 기반 자동 데이터 수집 }
        
        해당 기능은 딥러닝 모델 학습에 필요한 다량의 데이터를 빠르고 간편하게 수집하기 위해 개발되었음. 웹 크롤러를 통해 다양한 이미지 검색 엔진으로부터 동물 사진들을 자동으로 수집하고, 각각의 클래스마다 적절한 비율을 자동을 맞춰줌. 이를 통해 마일스톤을 2달 앞당길 수 있었으며, 해당 기능은 현재 제품에도 포함되어있음.
        
        \textbf{사용한 기술}: Python, Selenium
        
        \vspace{2mm}
        
        \textbf{1.1.2. 분산 컴퓨팅을 이용한 자동 신경망 학습}
        
        해당 기능은 수집한 데이터들을 다수의 GPU 장비에서 효율적이게 학습하기위해 개발되었음. 분산처리 프레임워크를 이용하여 효율적으로 데이터를 분산시켰고, 각각의 GPU 장비 내부에서 마이크로 스케줄링을 통해 학습 과정을 가속시켰음. 이를 통해 기존의 예상보다 2달 마일스톤을 앞당길 수 있었음.

        \textbf{사용한 기술}: Apache Spark, TFX, CUDA

        \vspace{2mm}
        
        \textbf{1.1.3. 모델 검증 자동화}

        해당 기능은 학습한 신경망의 정확도를 검증하는 작업을 자동화하기 위해 개발되었음. Jenkins와 Github Actions를 이용하여 CI/CD 파이프라인을 구성하였고, 모델 실행을 통해 딥러닝 서버를 연결하여 모델 검증을 자동으로 수행하는 인프라를 구축하였음. 이를 통해 최종적으로 96\% 정확도의 모델 개발을 할 수 있었음.
        
        \textbf{사용한 기술}: Jenkins, Github Actions, PyTorch
        
        \vspace{7mm}
        
        }

        \resumeSubProject{\textbf{1.2 [작업2]}}
        
        {
        [작업의 목표]를 목표로 둠. 구현한 기능들은 아래와 같음. 
        
        \vspace{2mm}
        
        \textbf{1.1.1. [기능1] }
        
        해당 기능은 [목적...]. [방법론...]. [결과와 임팩트...].
        
        \textbf{사용한 기술}: [기술1], [기술2]
        
        \vspace{2mm}
        
        \textbf{1.1.2. [기능2]}
        
        해당 기능은 [목적...]. [방법론...]. [결과와 임팩트...].

        \textbf{사용한 기술}: [기술1], [기술2], [기술3]

        \vspace{2mm}
        
        \textbf{1.1.3. [기능3]}

        해당 기능은 [목적...]. [방법론...]. [결과와 임팩트...].
        
        \textbf{사용한 기술}: [기술1], [기술2], [기술3]
        
        \vspace{7mm}

        }

        \resumeSubProject{\textbf{1.3 [작업3]}}
        
       {
        [작업의 목표]를 목표로 둠. 구현한 기능들은 아래와 같음. 
        
        \vspace{2mm}
        
        \textbf{1.1.1. [기능1] }
        
        해당 기능은 [목적...]. [방법론...]. [결과와 임팩트...].
        
        \textbf{사용한 기술}: [기술1], [기술2]
        
        \vspace{2mm}
        
        \textbf{1.1.2. [기능2]}
        
        해당 기능은 [목적...]. [방법론...]. [결과와 임팩트...].

        \textbf{사용한 기술}: [기술1], [기술2], [기술3]

        \vspace{2mm}
        
        \textbf{1.1.3. [기능3]}

        해당 기능은 [목적...]. [방법론...]. [결과와 임팩트...].
        
        \textbf{사용한 기술}: [기술1], [기술2], [기술3]
        
        \vspace{4mm}
        }

    \resumeSubProjectListEnd

  \resumeProjectListEnd

%-------------------------------------------
\section{\textbf{2. [프로젝트 이름] }}
  \resumeProjectListStart
    \resumeProject
    {\textbf{소속}: [회사 이름]} {2020.12 - PRESENT}
    
    {\textbf{주요 내용}:
    
    본 프로젝트는 [목적...]. [방법론...]. [결과와 임팩트...].
    
    \textbf{사용한 기술}: [기술1], [기술2], [기술3]
    
    }

  \resumeProjectListEnd

%-------------------------------------------
\section{\textbf{3. [프로젝트 이름}}
  \resumeProjectListStart
    \resumeProject
    {\textbf{소속}: [회사 이름]} {2020.12 - PRESENT}
    
    {\textbf{주요 내용}:
    
    본 프로젝트는 [목적...]. [방법론...]. [결과와 임팩트...].
    
    \textbf{사용한 기술}: [기술1], [기술2], [기술3]
    
    }
  \resumeProjectListEnd

%-------------------------------------------
\section{\textbf{4. [프로젝트 이름]}}
  \resumeProjectListStart
    \resumeProject
    {\textbf{소속}: [회사 이름]} {2020.12 - PRESENT}
    
    {\textbf{주요 내용}:
    
    본 프로젝트는 [목적...]. [방법론...]. [결과와 임팩트...].
    
    \textbf{사용한 기술}: [기술1], [기술2], [기술3]
    
    }

  \resumeProjectListEnd

%-------------------------------------------
\section{\textbf{5. [프로젝트 이름]}}
  \resumeProjectListStart
    \resumeProject
    {\textbf{소속}: [회사 이름]} {2020.12 - PRESENT}
    
    {\textbf{주요 내용}:
    
    본 프로젝트는 [목적...]. [방법론...]. [결과와 임팩트...].
    
    \textbf{사용한 기술}: [기술1], [기술2], [기술3]
    
    }

  \resumeProjectListEnd
  
\end{document}
